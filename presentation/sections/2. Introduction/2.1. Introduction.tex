
\section{Introdução}

\begin{frame}
  \frametitle{Introdução}

  \textbf{Imagine o processo de discussão de ideias em uma rede:}

  \begin{alertblock}{}
    \vspace{5mm}

    \begin{itemize}
      \item Como modelar a influência de determinados agentes?
      \item Como se dá a percepção de informações ao longo do tempo?
      \item Como o surgimento de novas informações afeta essa dinâmica?
    \end{itemize}
    \vspace{5mm}

  \end{alertblock}
\end{frame}

\begin{frame}
  \frametitle{Introdução: Modelo Proposto}

  \textbf{Elaboramos um modelo onde:}

  \begin{alertblock}{}
    \vspace{5mm}

    \begin{itemize}
      \item Cada nó da rede tem um valor no intervalo contínuo de [-1, 1];
      \item Uma ligação $(i, j)$ significa que $i$ recebe informação de $j$;
      \vspace{5mm}

      \item A influência exercida por um nó é proporcional ao número de nós que
        recebem informação dele.
    \end{itemize}
    \vspace{5mm}

  \end{alertblock}
\end{frame}

\begin{frame}
  \frametitle{Introdução: Modelo Proposto}

  \textbf{Função discreta para influência $x$ e posicionamento perante
  uma informação $y$:}

  \begin{alertblock}{}

    \begin{equation}
      x_{t}(i) = y_{t}(i) \  l_{t}^{r}(i),
    \end{equation}

    \begin{equation}
      y_{t+1}(i) = \frac{\sum_{j} \  x_{t}(j)}{\sum_{j} \  l_{t}^{r}(j)},
    \end{equation}

  \end{alertblock}
  \vspace{5mm}

  onde $l_{t}^{r}(i)$ são as ligações do grafo reverso do nó $i$.
\end{frame}

\begin{frame}
  \frametitle{Introdução: Modelo Proposto}

  \textbf{Podemos concluir então que:}

  \begin{alertblock}{}
    \vspace{5mm}

    \textbf{\alert{O posicionamento de um nó $i$ perante uma informação} é a
      média das influências dos nós adjacentes ponderadas pelas suas ligações
      no grafo reverso.}
    \vspace{5mm}

    \textbf{\alert{A influência de um nó $i$} é seu posicionamento perante uma
      informação multiplicada por suas ligações no grafo reverso.}
    \vspace{5mm}
  \end{alertblock}

\end{frame}
