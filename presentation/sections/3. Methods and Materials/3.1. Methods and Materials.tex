
\section{Materiais e Métodos}

\begin{frame}
  \frametitle{Materiais e Métodos}

  \begin{alertblock}{}
    \vspace{5mm}

    \begin{itemize}
      \item \alert{\textbf{O trabalho foi desenvolvido em \texttt{Python}}} e
        segue a estrutura de codificação adotada no livro texto de Hiroki Sayama
        \cite{Sayama:2015:ComplexSystems}.
      \vspace{5mm}

      \item \alert{\textbf{Utilizamos o módulo \texttt{PyCX}}} para facilitar
        a simulação e o acompanhamento dos resultados.
      \vspace{5mm}

      \item \textbf{Implementação referente à simulação} se encontra em
        \textbf{\texttt{/simulation/main.py}}.
      \vspace{5mm}

    \end{itemize}
  \end{alertblock}
\end{frame}

\begin{frame}
  \frametitle{Materiais e Métodos}

  \textbf{O código simulacional recebe três parâmetros:}

  \begin{alertblock}{}
    \vspace{5mm}

    \begin{itemize}
      \item \texttt{Path}: Arquivo \texttt{.gml} com rede a ser estudada
        (default = 11);
      \vspace{5mm}

      \item \texttt{TreatDataset}: \textit{flag} para sinalizar se os dados
        referentes à rede devem ser tratados (default = 0);
      \vspace{5mm}

      \item \texttt{SelfInfluence}: \textit{flag} para sinalizar se a
        auto-influência deve ser considerada (default = 0).
      \vspace{5mm}

    \end{itemize}
  \end{alertblock}
\end{frame}

\begin{frame}
  \frametitle{Materiais e Métodos}

  \textbf{O tratamento dos dados referentes à rede consiste em:}

  \begin{alertblock}{}
    \vspace{5mm}

    \begin{itemize}
      \item Remover nós com grau nulo, pois não são pertinentes para a dinâmica;
      \vspace{5mm}

      \item Transformar os valores dos nós para ponto flutuante;
      \vspace{5mm}

      \item Se o grafo contem valores -1 ou 0, transforma para -1.0.
      \vspace{5mm}

    \end{itemize}
  \end{alertblock}
\end{frame}
