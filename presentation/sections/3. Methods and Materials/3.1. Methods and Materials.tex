
\section{Materiais e Métodos}

\begin{frame}
  \frametitle{Materiais e Métodos}

  \begin{alertblock}{}
    \vspace{5mm}

    \begin{itemize}
      \item \alert{\textbf{O trabalho foi desenvolvido em \texttt{Python}}} e
        segue a estrutura de codificação adotada no livro texto de Hiroki Sayama
        \cite{Sayama:2015:ComplexSystems}.
      \vspace{5mm}

      \item \alert{\textbf{As sequências de números pseudo-aleatórios são
        geradas em \texttt{C++}}}, salvas em um arquivo texto, e lidas pelo
        programa códificado em \texttt{Python}.
      \vspace{5mm}

    \end{itemize}
  \end{alertblock}
\end{frame}

\begin{frame}
  \frametitle{Materiais e Métodos}

  \begin{alertblock}{}
    \vspace{5mm}

    \begin{itemize}
      \item \alert{\textbf{\texttt{Ranq2} \cite{Press:Numerical:Recipes}}} foi
        o algoritmo utilizado para gerar tais sequências.
      \vspace{5mm}

      \item \alert{\textbf{O método generalizado de Box-Müller foi empregado
        \cite{Tsallis:2007:GeneralizedBoxMuller}}} nos casos em que se
        transfomou a distribuição da sequência de números pseudo-aleatórios
        geradas.
      \vspace{5mm}

    \end{itemize}
  \end{alertblock}
\end{frame}
